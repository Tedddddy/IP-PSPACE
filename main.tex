\documentclass[hyperref={pdfpagelabels=false},t,10pt]{beamer}
\usepackage[utf8]{inputenc}
\usepackage[english]{babel}
\usepackage[T1]{fontenc}
\usepackage[default,scale=.95]{opensans}


\usetheme[cd2018]{tud}
\setbeamercolor{normal text}{fg=black}
\colorlet{alert}{cdblue}
\setbeamercolor{alerted text}{fg=cdblue}
\setbeamerfont{frametitle}{size=\Large,family=\sffamily,series=\sbseries}


\DeclareRobustCommand\sbseries{\fontseries{sb}\selectfont}
\DeclareTextFontCommand{\textsb}{\sbseries}


\title{IP = PSPACE}
\author[© author]{hello}
\institute{Technische Universit\"at Dresden}
\datecity{CONFERENCE}
\date{DATE}


\begin{document}

%%%% Uncomment the following line to set background image to main slide
%%%% (parameter sets transparency)
%\setbeamertemplate{tud background}[image/shaded]{background.jpg}{0.5}
\addtocounter{framenumber}{-1}
\maketitle

\begin{frame}
  \frametitle{PSPACE $\subseteq$ IP}
  For this inclusion, we use a well known PSPACE-complete problem, namely True-QBF
  \pause
  \begin{itemize}
  \item QBF-Truth is the set of all valid quantified boolean formulas without free variables and for any variable p we have $p\in \{0,1\}$
  \pause
  \item $\forall x \exists y \, (x \lor y)$, $\exists x \exists y \neg(x \land y)$
  \pause
  \item $\mbox{QBF-Truth}_{NNF}$ (abbrev. with QBF') is QBF-Truth but negations are only applied on variables
  \item $\exists x \exists y \neg(x\land y)$ is not in NNF but $\exists x \exists y (\neg x \lor \neg y)$
  \pause
  \item $\mbox{QBF} \leq^P_m \mbox{QBF-Truth}_{NNF}$ can be easily done by the verifier 
  \item it suffices to show QBF' $\in$ IP because we can reduce any problem in PSPACE to QBF in polynomial time \newline \newline
  How do we find an algorithm for QBF' s.t it satisfies the IP conditions?

  \end{itemize}
\end{frame}

\begin{frame}
  \frametitle{Arithmetization}
  The prover has to convince the verifier that the formula is valid but in case of 
  an invalid formula it should reject with high probability (for all prover) \newline \pause

  \begin{itemize}
    \item The idea is to arithmetize the formula \pause
    \item $x \land y$ becomes x*y
    \item $x \lor y$ becomes x+y - x*y
    \item $\neg x$ becomes 1-x \pause
  \end{itemize}
    \begin{tabular}{|c|c|c|c|c|}
      \hline
      x & y & $x\land y = x*y$& $x\land y = x+y - x*y $ &$\neg (x\land y) = 1- (x * y)$ \\ 
      \hline
      0 & 0 & 0 & 0 & 1\\ 
      0 & 1 & 0 & 1 & 1\\
      1 & 0 & 0 & 1 & 1\\
      1 & 1 & 1 & 1 & 0\\ 
    \end{tabular}
    $$\phi(x_1,x_2,...,x_n) = 1 \Leftrightarrow \phi_{arith}(x_1,x_2,...,x_n) = 1$$
\end{frame}

\begin{frame}
  \frametitle{Arithmetization}
  How do we arithmetize $\forall$ and $\exists$ ?  \newline
  \pause 

  \begin{itemize}
    \item $\forall x \, \phi$ becomes $a_0 * a_1$ where $a_0 = \phi[x := 0]$ and $a_1 = \phi[x := 1]$ 
    \pause
    \item $\exists x \, \phi$ becomes $a_0 + a_1$ where $a_0 = \phi[x := 0]$ and $a_1 = \phi[x := 1]$  
    \pause
  \end{itemize}
  Example : $\phi = \forall x \exists y \, \neg (x \land y)$ \newline \pause 
  Arithmetize : $\neg (x \land y) \, \xrightarrow{arith.} 1-(x * y) \xrightarrow{\exists arith.} \sum_{y \in \{0,1\}}^{} 1-(x*y) \xrightarrow{\forall arith.}$ \pause \newline 
  $\prod_{x \in \{0,1\}}^{} \sum_{y \in \{0,1\}} 1- (x*y) = \phi_{arith}$ \newline

  \begin{itemize}
    \item $\phi_{arith} = 2$ \pause
    \item If $\phi$ is true then $\phi_{arith} > 0$
    \item If $\phi$ is false then $\phi_{arith} = 0$ \pause
    \item This can be shown by structural induction 
  \end{itemize}
\end{frame}

\begin{frame}
  \frametitle{Arithmetization}
  \begin{itemize}
    \item $\phi = x$. If $\phi$ is true then $\phi_{arith} = 1$ and if $\phi $ is true then $\phi_{arith} = 0$
    \item $\phi = \neg x$ is the same but turned around \pause
    \item Suppose $\phi_1$ and $\phi_2$ are true, that means $\phi_{1arith} > 0$ and $\phi_{2arith} > 0$ \newline(it's similar for the other case)\pause
    \item $\phi_1 \land \phi_2$, $\phi_1 \lor \phi_2$ also holds
    \item For $\forall \, \phi_1$, we have by induction that $\phi_1[x := 0], \phi_1[x := 1]$ is true, so the multiplication of two positive value is positive (same for $\exists \phi_1$)
  \end{itemize}
\end{frame}

\begin{frame}
  \frametitle{Protocol-Problem}
  How do we start the communication between prover and verifier? \pause

  \begin{itemize}
    \item On input <$\phi$>, the prove sends a value c>0 to the verifier and tries to convince that c is the arithmetic value of $\phi$ \newline (Remember : the verifier can not calculate its value by itself because it could be double exponential) \pause
    \item Problem : the value could be exponential but the verifier has only polynomial time \pause
  \end{itemize}

  $\phi = \forall x_1 ... \forall x_m \exists y \exists z. (y \lor z)$. What is $\phi_{arith}$? We caluclate it step by step. \pause
  Let $\phi' = \exists y \exists z (y \lor z)$. Then $\phi'_{arith} = \sum_{y \{0,1\}}^{} \sum_{z \{0,1\}^{}} y + z - y*z = 3$ \pause
  $\phi_{arith} = \prod_{x_1 \in \{0,1\}}^{}... \prod_{x_m \in \{0,1\}}^{} \phi'_{arith} = 3^{2^{m}}$ \newline
  \begin{itemize}
    \item It holds that for formula $\phi$ with string length n : $\phi_{arith} \leq 2^{2^{n}}$
    \item  We solve this problem by using modulo with a suitable value
  \end{itemize}
\end{frame}

\begin{frame}
  \frametitle{Protocol-Problem}
  \begin{itemize}
    \item Pick a value $k \geq 2^n$ with two conditions : 
    \item k must be presentable in linear many bits \pause
    \item the calculation mod k must preserve ">0" for valid and "=0" for invalid formulas
    \item Number Theory Theorem : for any $a \leq 2^{2^{n}}, a>0$, there exist a prime number $k \in [2^n, 2^{3n}]$ s.t $a \not\equiv 0 $ (mod k)
  \end{itemize}
\end{frame}

\begin{frame}
  \frametitle{Protocol continued}
  \begin{itemize}
    \item Prover sends value c, prime number k and a proof b for the prime number property (it is possible to give a polynomial proof) \pause
    \item Verifier check c>0, $k \in [2^n, 2^{3n}]$ and b is a correct proof for prime property 
  \end{itemize}
  Even if k and b are correct, the verifier stays sceptical about c. \pause

  \begin{itemize}
    \item 
  \end{itemize}
  
\end{frame}




\end{document}
